\chapter{Objetives}\label{cap:objetives}

The realizacion of this practice has the main interest of analysing the
process of curing of an unsaturated polyester (UP) by measuring the
checmical composition of material using his absorption spectra.

The curing process, also named "polimerization" refers to the chemical reaction
that gives to the UP, strength, enabling the liquid polyester resin to become a
hard solid structure, due to the creation of crosslinks between the molecular chains
of polyester.

This is in theory possible due to the reduction in the total number of double bounds
present in the UP, which disapears in the cured
