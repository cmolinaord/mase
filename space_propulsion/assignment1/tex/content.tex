\LARGE{Assignment 1}. \Large{Space Propulsion}\\
\large{Carlos Molina Ordóñez}\\
Nov 20th, 2018

\section{Exercise 1}

Consider a spacecraft fitted with a reaction control system (RCS) consisting of thrusters
using Aerozine 50 fuel oxidized with nitrogen tetroxide, yielding a combustion
temperature $T_c = 3372 K$ and a gas with a molecular weight
$MW = 0.0226 kg/mol$ and adiabatic coefficient (specific heat ratio)
$\gamma = 1.24$. The combustion pressure is assumed to be $P_c = 210 MPa$,
and each thruster produces $F=100N$ of thrust.\\


The exit to throat area ratio $A_e/A_t$ has a value in a range from 50 to 100 (to be chosen by
the student), the following is requested:

\begin{enumerate}
	\item Propellant  mass flow
	\item Specific impulse
	\item Throat area
\end{enumerate}

\subsection{Solution}

From the equation:

\begin{equation}
	\frac{A_e}{A_t} = \frac{1}{M_e}{\left[\frac{2+(\gamma-1)M_e^2}{\gamma+1}\right]}^\frac{\gamma+1}{2(\gamma-1)}
\end{equation}

if we assume a given value for $\frac{A_e}{A_t} = 65$ we can compute the value of the Mach exit number $M_e$:
\begin{equation}
	M_e = 4.870
\end{equation}

With this Mach number we can compute the characteristic velocity $c^*$:

\begin{equation}
	c^* = \frac{1}{\sqrt{\gamma}}{\left(\frac{\gamma+1}{2}\right)}^\frac{\gamma+1}{2(\gamma-1)} \sqrt{R_g T_c}
\end{equation}

and, given that the gas constant is $\frac{R}{MW} = 367.9 ~ JK^{-1}kg^{-1}$, $c^*$ is:

\begin{equation}
	c^* = 1697.39
\end{equation}

Now, using that $c^*$ is also defined as:

\begin{equation}
	c^* = \frac{P_c A_t}{\dot{m}}
	\label{eq:c_star}
\end{equation}

we can obtain $\dot{m}$ if we know the area of the throat, which can be calculated,
after knowing the vacuum thrust coefficient in the next equation:

\begin{equation}
	C_F = \frac{F}{P_c A_t} = 1.8762
\end{equation}

Where the value of $C_F$ as a function of $\gamma$ and $M_e$ has been obtained
from the excel table. So, the value of the \textbf{throat area} can now be computed as:

\begin{equation}
	\boxed{A_t = 2.538·10^{-7} ~ m^2}
\end{equation}

And if we remember equation (\ref{eq:c_star}), we can now obtain the \textbf{propellant
mass flow}, $\dot{m}$:

\begin{equation}
	\boxed{\dot{m} = \frac{P_c A_t}{c^*} = 0.0314 ~ Kg·s^{-1}}
\end{equation}

And finally, given that the specific impulse, $I_{SP}$ is given by the equation:

\begin{equation}
	\boxed{I_{SP} = \frac{F}{\dot{m}g_0} = 324.6 ~ s}
\end{equation}
