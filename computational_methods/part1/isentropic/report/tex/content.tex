\begin{abstract}
In this exercise it's going to be studied a non-viscous irrotational fluid flows
from the computational point of view. The aim of the work is to study the equations
describing this type of fluids and how can we simulate them with a computer code.

The document will briefly explain the physics involved in the problem,
and how can it be transformed into the computer language. In a second part, the structure
of the code is explained. And finally, some results of the calculations
performed are shown, including some analysis for the computational performance.
\end{abstract}

\section{Physics of non-viscous fluids}

The physics governing fluid dynamics is complex, and in the general case it is
described by the Navier-Stokes equations. But as an approximation, the flow
around a certain object can be separated in two zones:
\begin{itemize}
	\item \textbf{Boundary layer:} is the region close to solid surfaces, where
	exists heat transfer and friction, which produces high gradients in speeds
	and temperatures.
	\item \textbf{Non-viscous regions:} the rest of the domain, where, as an
	approximation, the previous effects can be neglected.
\end{itemize}

In this study it's going to be studied this non-viscous case, considering that the
whole domain is under this assumption.




\begin{figure}[h]
	\centering
	%\includegraphics[width=0.8\textwidth]{img/}
	\caption[short caption]{Image caption}
	\label{fig:}
\end{figure}

Text\\

Reference \ref{fig:}\\
